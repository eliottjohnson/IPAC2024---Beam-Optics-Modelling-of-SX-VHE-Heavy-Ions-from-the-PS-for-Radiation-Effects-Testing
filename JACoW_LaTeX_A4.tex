% !TeX spellcheck = en_GB
% !TeX program = lualatex
%
% v 2.3  Feb 2019   Volker RW Schaa
%		# changes in the collaboration therefore updated file "jacow-collaboration.tex"
%		# all References with DOIs have their period/full stop before the DOI (after pp. or year)
%		# in the author/affiliation block all ZIP codes in square brackets removed as it was not %         understood as optional parameter and ZIP codes had bin put in brackets
%       # References to the current IPAC are changed to "IPAC'19, Melbourne, Australia"
%       # font for ‘url’ style changed to ‘newtxtt’ as it is easier to distinguish "O" and "0"
%
\documentclass[a4paper,
               %boxit,        % check whether paper is inside correct margins
               %titlepage,    % separate title page
               %refpage       % separate references
               biblatex,     % biblatex is used
               %keeplastbox,   % flushend option: not to un-indent last line in References
               %nospread,     % flushend option: do not fill with whitespace to balance columns
               %hyphens,      % allow \url to hyphenate at "-" (hyphens)
               %xetex,        % use XeLaTeX to process the file
               %luatex,       % use LuaLaTeX to process the file
               ]{jacow}
%
% ONLY FOR \footnote in table/tabular
%
\usepackage{pdfpages,multirow,ragged2e} %
%
% CHANGE SEQUENCE OF GRAPHICS EXTENSION TO BE EMBEDDED
% ----------------------------------------------------
% test for XeTeX where the sequence is by default eps-> pdf, jpg, png, pdf, ...
%    and the JACoW template provides JACpic2v3.eps and JACpic2v3.jpg which
%    might generates errors, therefore PNG and JPG first
%
\makeatletter%
	\ifboolexpr{bool{xetex}}
	 {\renewcommand{\Gin@extensions}{.pdf,%
	                    .png,.jpg,.bmp,.pict,.tif,.psd,.mac,.sga,.tga,.gif,%
	                    .eps,.ps,%
	                    }}{}
\makeatother

% CHECK FOR XeTeX/LuaTeX BEFORE DEFINING AN INPUT ENCODING
% --------------------------------------------------------
%   utf8  is default for XeTeX/LuaTeX
%   utf8  in LaTeX only realises a small portion of codes
%
\ifboolexpr{bool{xetex} or bool{luatex}} % test for XeTeX/LuaTeX
 {}                                      % input encoding is utf8 by default
 {\usepackage[utf8]{inputenc}}           % switch to utf8

\usepackage[USenglish]{babel}
\usepackage{hyperref}

\DeclareSIUnit\nucleon{u}
\DeclareSIUnit\torr{Torr}
\DeclareSIUnit\speedoflight{\text{c}}


%
% if BibLaTeX is used
%
\ifboolexpr{bool{jacowbiblatex}}%
 {%
  \addbibresource{references.bib}
 }{}



%%
%%   Lengths for the spaces in the title
%%   \setlength\titleblockstartskip{..}  %before title, default 3pt
%%   \setlength\titleblockmiddleskip{..} %between title + author, default 1em
%%   \setlength\titleblockendskip{..}    %afterauthor, default 1em

\begin{document}

\title{Beam Optics Modelling of Slow-Extracted Very High-Energy Heavy Ions from the CERN Proton Synchrotron for Radiation Effects Testing \thanks{The HEARTS project is funded by the European Union under Grant Agreement No. 101082402, through the Space Work Programme of the European Commission.}}

\author{E. P. Johnson\thanks{eliott.philippe.johnson@cern.ch}, M.A. Fraser, CERN, Geneva, CH}
	
\maketitle

%
\begin{abstract}
   Testing of space-bound microelectronics plays a crucial role in ensuring the reliability of electronics exposed to the challenging radiation environment of outer space. This contribution describes the beam optics studies carried out for the run held in November 2023 in the context of the CERN High-Energy Accelerators for Radiation Testing and Shielding (HEARTS) experiment. It also delves into an investigation of the initial conditions at the start of the transfer line from the CERN Proton Synchrotron (PS) to the CERN High Energy Accelerator Mixed-field (CHARM) facility. Comprehensive optics measurement and simulation campaigns were carried out for this purpose and are presented here. Using a validated optics model of the transfer line, the impact of air scattering on the beam size was quantified with MAD-X and FLUKA, providing valuable insights into the current performance and limitations for Single Event Effects (SEE) testing at CHARM.
\end{abstract}







%%%%%%%%%%%%%%%%%%%%%%%%%%%%
%%%% OPTICS MEASUREMENT %%%%
%%%%%%%%%%%%%%%%%%%%%%%%%%%%
\section{OPTICS MODEL}

CERN High-Energy Accelerators for Radiation Testing and Shielding (HEARTS) \cite{noauthor_hearts_nodate} use slow-extracted beams from CERN's Proton Synchrotron (PS) via the F61 and T8 transfer lines to irradiate Devices Under Test (DUT) in CHARM (CERN High Energy Accelerator Mixed-field facility). The irradiation of electronic components requires an accurate beam model to predict the beam size on the target. This section will cover the beam optics measurements performed on the 2 GeV/u lead ion beam (\( \text{Pb}^{54+} \)) used to improve the optics model.

\subsection{Quadrupole scan measurement}

During parallel Machine Development (MD) studies, quadrupole scans were conducted on the PS to East Dump transfer line to establish the initial beam conditions post-extraction. This transfer line includes three quadrupoles and a Beam Television (BTV) equipped with a fluorescent screen for beam size measurements. The scans, conducted at the upper limits of the energy range provided by HEARTS, aimed to mitigate the impact of beam-material interaction. Py-BOBYQA \cite{cartis_escaping_2022, cartis_improving_2019} was chosen as the optimization algorithm to minimize the sum of squared differences between the simulated and measured beam sizes. The optimizer was used to determine a set of initial Courant-Snyder (Twiss) parameters and normalized emittances while holding the dispersion fixed, to match the measurements.


\subsection{Dispersion measurement}

In a second MD, the dispersion was measured in the T8 transfer line at two different BTV locations (T08.BTV020, T08.BTV035) and at the Multi-Wire Proportional Chamber (MWPC) located at the end of the line (T08.MWPC). The procedure involved changing the beam's momentum through the revolution frequency—the rate at which particles complete one full orbit around the accelerator path—and measuring the subsequent centroid movement, as the dispersion is given by:

\begin{equation}
D = \frac{\Delta x}{\frac{\Delta p}{p}}
\end{equation}

where $\Delta x$ is the change in the centroid position, and $\frac{\Delta p}{p}$ is the relative change in the beam's momentum.

The change in revolution frequency and the relative change in beam momentum can be expressed as:

\begin{equation}
\frac{\Delta f}{f} = -\eta \frac{\Delta p}{p}
\end{equation}

Here, $\eta$ is the slip factor, defined as:

\begin{equation}
\eta = \left(\frac{1}{\gamma_{\text{tr}}^{2}} - \frac{1}{\gamma^{2}}\right)
\end{equation}

where $\gamma = 11.414$ is the relativistic Lorentz factor for the beam, and $\gamma_{\text{tr}} = 6.12719$ is the transition gamma. These were determined using the energy of the ion beam and a MAD-X \cite{noauthor_mad_nodate} model of the PS. The dispersion is calculated from the slopes of the measured changes in centroid position versus the relative change in momentum, corrected by the factor of $\frac{1}{\eta f_{\text{rev}_0}}$ where the initial revolution frequency, $f_{\text{rev}_0}$, is $452.0$ kHz. The dispersion measurements are presented in Table \ref{tab:dispersion}. A caveat of this measurement is that it was performed for only one optic. A more rigorous method involves adjusting the revolution frequency ($f_{\text{rev}}$) for each optics.

\begin{table}[h!]
\centering
\caption{Dispersion Measurements}
\begin{tabular}{l c c}
\hline
Device & \(Dx\) & \(Dy\) \\
\hline
BTV020  & \(-0.197 \pm 0.034\) & \(0.035 \pm 0.031\) \\
BTV035  & \(0.851 \pm 0.047\) & \(-0.061 \pm 0.009\) \\
MWPC   & \(-0.963 \pm 0.082\) & \(-0.169 \pm 0.019\) \\
\hline
\label{tab:dispersion}
\end{tabular}
\end{table}

\begin{figure}[!htb]
   \centering
   \includegraphics*[width=1.0\columnwidth]{dispersion_diff.png}
   \caption{Difference in dispersion before and after the re-match.}
   \label{fig:dispersion}
\end{figure}

A match of the dispersion at the start of the F61 line was done using these measurement and the optimiser was once again ran on the Courant-Synder parameters as they collectively influence the beam size. The new set of initial conditions (including the new fixed dispersion) is presented in Table \ref{tab:initial_conditions_comparison}. An empirical model has proven to be more reliable that a stitched model through the highly non-linear stray fields from the PS main units \cite{johnson_beam_2022}.


\begin{table}[h!]
    \centering
    \caption{Comparison of Matched Initial Parameters}
    \label{tab:initial_conditions_comparison}
    \begin{tabular}{
    l 
    S[table-format=2.3] 
    @{${}\rightarrow{}$} 
    S[table-format=2.3]
    }
        \hline
        {Parameter} & \multicolumn{2}{c}{Initial \& Re-matched} \\
        \hline
        {$\beta_x$ (m)} & 53.074 & 66.748 \\
        {$\beta_y$ (m)} & 3.675 & 3.764 \\
        {$\alpha_x$} & -13.191 & -16.272 \\
        {$\alpha_y$} & 0.859 & 0.703 \\
        {$D_x$ (m)} & 0.13 & 0.086 \\
        {$D_y$ (m)} & 0.0 & -0.003 \\
        {$D'_x$} & 0.02 & 0.017 \\
        {$D'_y$} & 0.0 & -0.005 \\
        {$\varepsilon_{nx}$ (\si{\metre^{-5}})} & 2.53e-5 & 2.28e-5 \\
        {$\varepsilon_{ny}$ (\si{\metre^{-5}})} & 6.94e-6 & 8.63e-6 \\
        {$\frac{\sigma_{E}}{E}$} & \multicolumn{2}{c}{0.0045} \\
        \hline
    \end{tabular}
\end{table}




%%%%%%%%%%%%%%%%%%%%%%%%%%%%%%%%
%%%% AIR SCATTERING %%%%%%%%%%%%
%%%%%%%%%%%%%%%%%%%%%%%%%%%%%%%%
\section{Air scattering}


From the extraction of the PS to the DUT, several tens of meters are crossed by the beam which blows-up the emittance and straggles the beam's energy through matter interaction. The main effects are an increase in beam size and a change in the beam's rigidity. MAD-X does not include beam matter interaction, thus a custom multi-coulomb scattering code has been implemented \footnote{An example is available in the \href{https://gitlab.cern.ch/abt-optics-and-code-repository/simulation-codes/pybt/-/blob/master/pybt/examples/example_air_scattering.ipynb}{PyBT GitLab repository}.} to add this effect in the simulation and try to reproduce the blow up in beam size. The angle of deflection due to multi-Coulomb scattering as a particle passes through a material is given by  \cite{muller_description_2001}:

\[
\theta_{rms} = \frac{\SI{13.6}{\mega\electronvolt\per\speedoflight}}{p\beta_{p}}q_{p}\sqrt\frac{L}{L_{rad}}\]

where $p$ is the beam's total energy in MeV, $q_{p}$ the number of charges, $P=\SI{1.013}{\bar}$ is the standard air pressure at sea level, $P_{\text{Torr}}=P\cdot\SI{750.062}{\torr}$ is the air pressure, $L_{\text{rad0}}=\SI{301}{}$ is the radiation length for air \footnote{\href{https://cds.cern.ch/record/941314/files/p245.pdf}{{https://cds.cern.ch/record/941314/files/p245.pdf}}}, and $L_{\text{rad}}$ is calculated as $L_{\text{rad0}}/(P_{\text{Torr}}/760)$ to adjust for the actual air pressure and $L$ is the length of the interaction. For the Twiss parameters and emittance change the following equations result:


\begin{equation}
\label{eq:mc_eq}
\begin{aligned}
\alpha_{1} &= \frac{\epsilon_0 \alpha_0 - \frac{L}{2} \theta_{\text{rms}}^2}{\epsilon_0 + \Delta \epsilon}\\
\beta_{1} &= \frac{\epsilon_0 \beta_0 + \frac{L^2}{3} \theta_{\text{rms}}^2}{\epsilon_0 + \Delta \epsilon}\\
\epsilon_{1} &= \epsilon_0 + \frac{1}{2} \theta_{\text{rms}}^2 \left( \beta_0 + L \alpha_0 + \frac{L^2}{3} \gamma_0 \right)
\end{aligned}
\end{equation}

The modular design of the code allows for effective handling of multi-Coulomb interactions in any MAD-X sequence. After loading the MAD-X sequence, the user introduces air regions along the sequence with a specified integration length $L$ in meters. The process\_scattering() function is called to calculate the beam's parameters post multi-Coulomb scattering, generating an updated Twiss output. Upon reaching the "AIR\_START" marker, the function records the beta functions using "SAVEBETA" and segments the beam line at this point. It then conducts a Twiss calculation with the saved values, updating the beta functions and emittances with Equations~\ref{eq:mc_eq}. This procedure is repeated at each "INNER\_MARKER," refining the beam optics iteratively to account for multi-Coulomb scattering effects in air regions.

\begin{figure}[!htb]
   \centering
   \includegraphics*[width=1.0\columnwidth]{air_scattering.png}
   \caption{Toy lattice showcasing analytical simulation of multi-coulomb scattering.}
   \label{fig:simple_line}
\end{figure}

Analytical results were benchmarked against Monte Carlo simulations that applied random transverse displacements to particle trajectories, simulating scattering with a Gaussian-distributed angle $\theta_{RMS}$. This process emulated the randomness of actual scattering, generating a variety of particle paths to compare statistical beam properties between both methods. Table~\ref{tab:sigma_comparison} contrasts these simulation approaches, noting that while the Monte Carlo method requires 30 minutes for 50,000 particles, the analytical method takes only 1-2 seconds, irrespective of particle count.

\begin{table}[ht]
\centering
\resizebox{\columnwidth}{!}{% Adjusts the table to column width
\begin{tabular}{lcc}
\hline
Distribution & $\sigma$ Analytical (m) & $\sigma$ Monte Carlo (m) \\
\hline
H Non-Scattered & 7.41e-03 & 7.43e-03 \\
H Scattered     & 1.05e-02 & 1.03e-02 \\
V Non-Scattered   & 4.53e-03 & 4.55e-03 \\
V Scattered       & 8.67e-03 & 8.45e-03 \\
\hline
\end{tabular}
}
\caption{Simulated beam size for analytical and Monte Carlo methods in non-scattered and scattered particle distributions.}
\label{tab:sigma_comparison}
\end{table}

The current HEARTS setup encounters limitations due to extensive air regions along the beam path. One straightforward improvement would be to shift irradiation from CHARM to IRRAD, which is located just a few meters upstream, minimizing air travel. Additionally, installing a vacuum chamber in the switching dipole of F61 could significantly streamline operations.






%%%%%%%%%%%%%%%%%%%%%%%%%%%%%%%%
%%%% ENERGY CONTROL %%%%%%%%%%%%
%%%%%%%%%%%%%%%%%%%%%%%%%%%%%%%%
\section{Energy control}

Rapid beam energy adjustments, essential for exploring various Linear Energy Transfer (LET) regions and penetration depths, are facilitated by a process developed by CHIMERA at CERN. This involves adjusting the magnetic field at the PS's flat top to change beam rigidity. The makerule algorithm recalculates the magnetic field needed for transfer line magnets, and removing Pole Face Windings (PFW), which don't scale linearly with rigidity, ensures smooth energy variation. In 2023, the use of distinct cycles for each energy level was replaced by a unified cycle managed by a Python script. This new setup allows for quick, seamless transitions between energy levels—\SI{650}{}, \SI{750}{}, and \SI{1000}{\mega\electronvolt\per\nucleon}—executing changes every \SI{15}-\SI{30}{\second}. The unified cycle also supports energy scans that determine the beam's kinetic energy using penetration tests in materials like PolyMethyl Methacrylate (PMMA) and speeds up Single Event Effects (SEE) testing of components \cite{noauthor_hearts_nodate}. Additionally, Python scripting provides control over fluence—the total ion count impacting the DUT—which is vital for accurate radiation effect testing. A script ensures the beam is disabled once the target fluence is achieved, with fluence measured by the calibrated XSEC070, which utilizes its single ion counting capability.

\subsection{Straggling effects}
Energy straggling affects the beam's kinetic energy as it traverses air regions in the F61 and T08 transfer lines. This effect, modeled by FLUKA simulations which include vacuum windows and beam instrumentation \cite{battistoni_overview_2015}, can be measured using penetration range measurement with different degraders thickness. As shown in Figure \ref{fig:straggling_effects}, at T08.BTV35, lower kinetic energy at extraction leads to more significant beam displacement, confirming FLUKA's predictions. Installing vacuum pipes, particularly in key areas such as F61.MBXHD025, is recommended to reduce air-induced straggling. This would optimize VHE ion irradiation at IRRAD/CHARM by improving beam transport.

\begin{figure}[!htb]
   \centering
   \includegraphics*[width=0.9\columnwidth]{straggling_effects.png}
   \caption{Measurement of the straggling effects of air in the T8 line at BTV035 compared FLUKA simulation.}
   \label{fig:straggling_effects}
\end{figure}




%%%%%%%%%%%%%%%%%%%%%%%%%%%%%%%%%%%%%%%%%%%%%%%%%
%%%% COMPARISON MEAS AND SIM %%%%
%%%%%%%%%%%%%%%%%%%%%%%%%%%%%%%%%%%%%%%%%%%%%%%%%
\section{Comparison of measurement to simulation}
% This section presents a comparison between the measured beam optics at various points along the transfer line and the predictions made by our MCS-enhanced simulation model.

% Optics measurements were performed using the Beam Television (BTV) monitors along the transfer line. These measurements include profiles at various beam energies, extracted using the slow-extraction process from the CERN PS. Simultaneously, an enhanced model incorporating MCS effects was developed using the MAD-X simulation tool, adjusted to account for the scattering parameters calculated using the following MCS code discussed previously.

% \subsection{Results and Discussion}
% The results from the BTV measurements and MAD-X simulations are compiled in Table \ref{tab:optics_comparison}. The table lists the measured and simulated beam sizes, showing both the horizontal and vertical measurements across different stations.

%\subsection{Comparison with MWPC}


Figure~\ref{fig:diff_beam_size} illustrates the comparison between measured and simulated horizontal (H) and vertical (V) beam sizes as a function of quadrupole strength. The plot shows that the vertical beam measurements exhibit increased errors due to the beam's interaction with the vacuum pipe aperture, which alters its Gaussian shape. The aperture scraping may occur due to the varying dipolar moment experienced during quadrupole scans, as the beam does not travel through the center of the quadrupoles. This comparison validates the accuracy of the simulations used for single event effect (SEE) testing at CHARM. 

\begin{figure}[!htb]
   \centering
   \includegraphics*[width=0.9\columnwidth]{beam_size_comparison.png}
   \caption{Comparison of beam size measurements and simulation.}
   \label{fig:diff_beam_size}
\end{figure}


%\subsection{Comparison with Octavius array}


%During the final day of the November 2023 CHIMERA HEARTS run, RP conducted a survey with the beam energy set at 3 GeV/n. Below is a figure illustrating the comparison between the MAD-X model and FLUKA predictions as well as measurements taken by the MWPC.

%\begin{figure}[!htb]
%\centering
%\includegraphics*[width=1.0\columnwidth]{rp_survey.png}
%\caption{Difference in beam size between measurement and simulation.}
%\label{fig:diff_beam_size}
%\end{figure}



%\subsection{Comparison with BTV}
%\subsection{Mention difficulty of changing the optics because the beam is not well centered}

%\section{Beam size as a function of RFKO gain}
%\subsection{I've observed that the emittance changes at different energy.}
%\subsection{Link with Wesley's contribution}






%%%%%%%%%%%%%%%%%%%%%%%%%%%%%%%%
%%%% CONCLUSION %%%%%%%%%%%%%%%%
%%%%%%%%%%%%%%%%%%%%%%%%%%%%%%%%
\section{CONCLUSION}

This study has advanced the beam optics modeling for radiation effects testing with slow-extracted VHE heavy ions from the PS. Key advancements include the refined measurement of initial conditions, improved understanding of dispersion effects, and the quantification of air scattering effects. The validated beam model significantly enhances the precision and reliability of radiation testing setups. Future improvements will focus on moving from CHARM to IRRAD and additional vacuum section to minimize air interactions and further refine beam control and homogeneity. These contributions not only enhance the robustness of radiation testing at CERN but also provide valuable insights for similar facilities globally.


\ifboolexpr{bool{jacowbiblatex}}%
	{\printbibliography}%
	{%
	% "biblatex" is not used, go the "manual" way
	
	%\begin{thebibliography}{99}   % Use for  10-99  references
	\begin{thebibliography}{9} % Use for 1-9 references
	
	\bibitem{jacow-help}
		JACoW,
		\url{http://www.jacow.org}
	
	\bibitem{IEEE}
		\textit{IEEE Editorial Style Manual},
		IEEE Periodicals, Piscataway,
		NJ, USA, Oct. 2014, pp. 34--52.

	\bibitem{journal-abbreviations}
	\url{https://woodward.library.ubc.ca/researchhelp/journal-abbreviations/}

	\end{thebibliography}
} % end \ifboolexpr

\end{document}
