% !TeX spellcheck = en_GB
% !TeX program = lualatex
%
% v 2.3  Feb 2019   Volker RW Schaa
%		# changes in the collaboration therefore updated file "jacow-collaboration.tex"
%		# all References with DOIs have their period/full stop before the DOI (after pp. or year)
%		# in the author/affiliation block all ZIP codes in square brackets removed as it was not %         understood as optional parameter and ZIP codes had bin put in brackets
%       # References to the current IPAC are changed to "IPAC'19, Melbourne, Australia"
%       # font for ‘url’ style changed to ‘newtxtt’ as it is easier to distinguish "O" and "0"
%
\documentclass[a4paper,
               %boxit,        % check whether paper is inside correct margins
               %titlepage,    % separate title page
               %refpage       % separate references
               biblatex,     % biblatex is used
               %keeplastbox,   % flushend option: not to un-indent last line in References
               %nospread,     % flushend option: do not fill with whitespace to balance columns
               %hyphens,      % allow \url to hyphenate at "-" (hyphens)
               %xetex,        % use XeLaTeX to process the file
               %luatex,       % use LuaLaTeX to process the file
               ]{jacow}
%
% ONLY FOR \footnote in table/tabular
%
\usepackage{pdfpages,multirow,ragged2e} %
%
% CHANGE SEQUENCE OF GRAPHICS EXTENSION TO BE EMBEDDED
% ----------------------------------------------------
% test for XeTeX where the sequence is by default eps-> pdf, jpg, png, pdf, ...
%    and the JACoW template provides JACpic2v3.eps and JACpic2v3.jpg which
%    might generates errors, therefore PNG and JPG first
%
\makeatletter%
	\ifboolexpr{bool{xetex}}
	 {\renewcommand{\Gin@extensions}{.pdf,%
	                    .png,.jpg,.bmp,.pict,.tif,.psd,.mac,.sga,.tga,.gif,%
	                    .eps,.ps,%
	                    }}{}
\makeatother

% CHECK FOR XeTeX/LuaTeX BEFORE DEFINING AN INPUT ENCODING
% --------------------------------------------------------
%   utf8  is default for XeTeX/LuaTeX
%   utf8  in LaTeX only realises a small portion of codes
%
\ifboolexpr{bool{xetex} or bool{luatex}} % test for XeTeX/LuaTeX
 {}                                      % input encoding is utf8 by default
 {\usepackage[utf8]{inputenc}}           % switch to utf8

\usepackage[USenglish]{babel}
\usepackage{hyperref}

\DeclareSIUnit\nucleon{u}
\DeclareSIUnit\torr{Torr}

%
% if BibLaTeX is used
%
\ifboolexpr{bool{jacowbiblatex}}%
 {%
  \addbibresource{references.bib}
 }{}



%%
%%   Lengths for the spaces in the title
%%   \setlength\titleblockstartskip{..}  %before title, default 3pt
%%   \setlength\titleblockmiddleskip{..} %between title + author, default 1em
%%   \setlength\titleblockendskip{..}    %afterauthor, default 1em

\begin{document}

\title{Beam Optics Modelling of Slow-Extracted Very High-Energy Heavy Ions from the CERN Proton Synchrotron for Radiation Effects Testing \thanks{The HEARTS project is funded by the European Union under Grant Agreement No. 101082402, through the Space Work Programme of the European Commission.}}

\author{E. P. Johnson\thanks{eliott.philippe.johnson@cern.ch}, M.A. Fraser, CERN, Geneva, CH}
	
\maketitle

%
\begin{abstract}
   Testing of space-bound microelectronics plays a crucial role in ensuring the reliability of electronics exposed to the challenging radiation environment of outer space. This contribution describes the beam optics studies carried out for the run held in November 2023 in the context of the CERN High-Energy Accelerators for Radiation Testing and Shielding (HEARTS) experiment. It also delves into an investigation of the initial conditions at the start of the transfer line from the CERN Proton Synchrotron to the CERN High Energy Accelerator Mixed-field (CHARM) facility. Comprehensive optics measurement and simulation campaigns were carried out for this purpose and are presented here. Using a validated optics model of the transfer line, the impact of air scattering on the beam size was quantified with MAD-X and FLUKA, providing valuable insights into the current performance and limitations for Single Event Effects (SEE) testing at CHARM.
\end{abstract}







%%%%%%%%%%%%%%%%%%%%%%%%%%%%
%%%% OPTICS MEASUREMENT %%%%
%%%%%%%%%%%%%%%%%%%%%%%%%%%%
\section{OPTICS MODEL}

HEARTS uses slow-extracted beams from the PS via the F61 and T8 transfer lines to irradiate Devices Under Test (DUT) in CHARM (CERN High Energy Accelerator Mixed-field facility) which requires an accurate beam model. This section will go through beam optics measurement performed on the 2 GeV/u lead ion beam used to improve the optics model.

\subsection{Quadrupole scan measurement}

 Quadrupole scans were performed, during parallel Machine Development (MD) studies, using the PS to East Dump transfer line to match the initial conditions of the beam after extraction. This transfer line includes three quadrupoles and a Beam Television (BTV) using a fluorescent screen for beam size measurements. These scans were conducted at the upper limits of the energy range provided by HEARTS to mitigate the impact of beam-material interaction. These measurements were used to create a set of initial Courant–Snyder (Twiss) parameters, and normalized emittances whilst holding the dispersion fixed. Py-BOBYQA \cite{cartis_escaping_2022, cartis_improving_2019} was chosen as the optimization algorithm to minimize the sum of squared differences between the simulated and measured beam sizes.


\subsection{Dispersion measurement}

In a separate step, the dispersion was measured in the T8 transfer line at two different BTV locations (T08.BTV020, T08.BTV035) and at the Multi-Wire Proportional Chamber (MWPC) located at the end of the line (T08.MWPC). The procedure involved changing the beam's momentum through the revolution frequency and measuring the subsequent centroid movement as the dispersion is given by:

\begin{equation}
D = \frac{\Delta x}{\frac{\Delta p}{p}}
\end{equation}

where $\Delta x$ is the change in the centroid position, and $\frac{\Delta p}{p}$ is the relative change in the beam's momentum.

The change in revolution frequency and the relative change in beam momentum can be expressed as:

\begin{equation}
\frac{\Delta f}{f} = -\eta \frac{\Delta p}{p}
\end{equation}

Here, $\eta$ is the slip factor, calculated using:

\begin{equation}
\eta = \left(\frac{1}{\gamma_{\text{tr}}^{2}} - \frac{1}{\gamma^{2}}\right)
\end{equation}

where $\gamma = 11.414$ is the relativistic Lorentz factor for the beam, and $\gamma_{\text{tr}} = 6.12719$ is the transition gamma which were found using the energy of the ion beam and a MAD-X model. The dispersion measurement are calculated from the slope of the measured change in centroid position versus the relative change in momentum, corrected by the factor of $\frac{1}{\eta f_{\text{rev}_0}}$ where the revolution frequency, $f_{\text{rev}_0}$, is initially $452.0$ kHz and are presented in Table \ref{tab:dispersion}.

% \begin{figure}[!htb]
%    \centering
%    \includegraphics*[width=1.0\columnwidth]{beam_size_diff.png}
%    \caption{Difference in beam size between measurement and simulation.}
%    \label{fig:diff_beam_size}
% \end{figure}

\begin{table}[h!]
\centering
\caption{Dispersion Measurements}
\begin{tabular}{l c c}
\hline
Device & \(Dx\) & \(Dy\) \\
\hline
BTV020  & \(-0.197 \pm 0.034\) & \(0.035 \pm 0.031\) \\
BTV035  & \(0.851 \pm 0.047\) & \(-0.061 \pm 0.009\) \\
MWPC   & \(-0.963 \pm 0.082\) & \(-0.169 \pm 0.019\) \\
\hline
\label{tab:dispersion}
\end{tabular}
\end{table}


\begin{figure}[!htb]
   \centering
   \includegraphics*[width=1.0\columnwidth]{dispersion_diff.png}
   \caption{Difference in dispersion before and after the rematch.}
   \label{fig:dispersion}
\end{figure}

A rematching of the Courant-Snyder parameters was done using these initial dispersion values as they collectively influence the beam siz.

\subsection{Transfer Line Initial Conditions}

However, the initial approach contained a limitation in that the dispersion correction was performed solely for one optical setting. A more rigorous method would entail adjusting the revolution frequency ($f_{\text{rev}}$) for each optical configuration to precisely tailor the dispersion characteristics throughout the beam line.

\begin{table}[h!]
    \centering
    \caption{Comparison of Matched Initial Parameters}
    \label{tab:initial_conditions_comparison_single_col}
    \begin{tabular}{l c}
        \hline
        Parameter & Initial \(\rightarrow\) Rematched \\
        \hline
        $\beta_x$ (m) & 53.074 \(\rightarrow\) 66.748 \\
        $\beta_y$ (m) & 3.675 \(\rightarrow\) 3.764 \\
        $\alpha_x$ & -13.191 \(\rightarrow\) -16.272 \\
        $\alpha_y$ & 0.859 \(\rightarrow\) 0.703 \\
        $D_x$ (m) & 0.13 \(\rightarrow\) 0.086 \\
        $D_y$ (m) & 0.0 \(\rightarrow\) -0.003 \\
        $D'_x$ & 0.02 \(\rightarrow\) 0.017 \\
        $D'_y$ & 0.0 \(\rightarrow\) -0.005 \\
        $\varepsilon_{nx}$ ($\text{m}^{-5}$) & $2.53 \times 10^{-5}$ \(\rightarrow\) $2.28 \times 10^{-5}$ \\
        $\varepsilon_{ny}$ ($\text{m}^{-5}$) & $6.94 \times 10^{-6}$ \(\rightarrow\) $8.63 \times 10^{-6}$ \\
        $\frac{\sigma_{E}}{E}$ & 0.0045 \\
        \hline
    \end{tabular}
\end{table}


Stray fields from the PS main units are challenging as has been shown in a previous IPAC proceeding thus empirical measurements are preferred.






%%%%%%%%%%%%%%%%%%%%%%%%%%%%%%%%
%%%% AIR SCATTERING %%%%%%%%%%%%
%%%%%%%%%%%%%%%%%%%%%%%%%%%%%%%%
\section{Air scattering}

In particle physics, accurately predicting beam size in accelerators is essential but challenging due to multi-Coulomb scattering. This phenomenon, where charged particles interact with atomic nuclei, significantly alters beam trajectory and spread, deviating from idealized simulations. Ignoring these interactions leads to inaccuracies in beam focus and efficiency, impacting experiments and applications like medical therapies. Incorporating multi-Coulomb scattering into simulations enhances the precision of beam behavior predictions, facilitating better experiment design and accelerator functionality.

\subsection{Overview of the formula}

The fundamental formula that describes the angle of deflection due to multi-Coulomb scattering as a particle passes through a material is given by the following formula:

\[
\theta_{rms} = \frac{13.6 \text{MeV/c}}{p\beta_{p}}q_{p}\sqrt\frac{L}{L_{rad}}\]

where $p$ is the beam's total energy in MeV, $q_{p}$ the number of charges, $P = \SI{1.01325}{\bar}$ is the standard air pressure at sea level, $P_{\text{Torr}} = P \times$ \SI{750.062}{\torr} is the air pressure, $L_{\text{rad0}} = \SI{301}{}$ is the radiation length for air \footnote{\href{https://cds.cern.ch/record/941314/files/p245.pdf}{{https://cds.cern.ch/record/941314/files/p245.pdf}}}, and $L_{\text{rad}}$ is calculated as $L_{\text{rad0}}/(P_{\text{Torr}}/760)$ to adjust for the actual air pressure and $L$ is the length of the interaction.


\begin{equation}
\begin{aligned}
\alpha_{1} &= \frac{\epsilon_0 \alpha_0 - \frac{L}{2} \theta_{\text{rms}}^2}{\epsilon_0 + \Delta \epsilon}\\
\beta_{1} &= \frac{\epsilon_0 \beta_0 + \frac{L^2}{3} \theta_{\text{rms}}^2}{\epsilon_0 + \Delta \epsilon}\\
\epsilon_{1} &= \epsilon_0 + \frac{1}{2} \theta_{\text{rms}}^2 \left( \beta_0 + L \alpha_0 + \frac{L^2}{3} \gamma_0 \right)
\end{aligned}
\end{equation}


\subsection{Overview of the script}

The concept behind the script is designed to handle multi-coulomb interactions by integrating a straightforward debugging process with optimization through pybobqa. It introduces a user interaction mechanism where the user defines the boundaries of an air region with markers. Additionally, users can specify the step size for computations, which can vary from 1cm to 1m as well as an arbitrary number of air regions. The core functionality of the script includes adding inner markers within these air regions at discrete, user-selected intervals for performing calculations. At the onset of the first "AIR\_START" marker, the script is programmed to save the beta function values using "SAVEBETA" and then proceeds to split the sequence at "AIR\_START." Subsequently, it runs a Twiss calculation on this split sequence, utilizing the saved beta function values. However, it updates the beta functions for both the horizontal ( $\beta_{x}$ ) and vertical ( $\beta_{y}$ ) planes, as well as the horizontal and vertical emittances ( $e_{x}$, $e_{y}$ ), based on the calculations. The process continues in a loop: running Twiss calculations up to the next "INNER\_MARKER," saving the beta function values again with "SAVEBETA," and then splitting the sequence, thereby allowing for precise and iterative adjustments based on multi-coulomb interactions within specified air regions.

\begin{figure}[!htb]
   \centering
   \includegraphics*[width=1.0\columnwidth]{air_scattering.png}
   \caption{Toy lattice showcasing simulation of multi-coulomb scattering using the analytical formula.}
   \label{fig:simple_line}
\end{figure}

The analytical computation was compared to a Monte Carlo simulation where each particle's trajectory through the lattice was perturbed to simulate the effects of scattering phenomena. This was achieved by introducing random transverse kicks at predefined lattice positions. The amplitude of these stochastic perturbations was governed by a Gaussian distribution with a zero mean and a standard deviation defined by $\theta_{RMS}$, representing the root-mean-square scattering angle. This procedure models the random nature of particle interactions within the medium, which is a characteristic aspect of the physical scattering process. Each particle was subjected to this randomization process, thereby generating a distribution of trajectories that collectively represent the beam's transverse profile post-scattering. The data accrued from these simulations provided an ensemble of particle paths, which were subsequently used to elucidate the statistical properties of the beam, such as the spread in the particle distribution, and to compare the Monte Carlo results against the analytical predictions, thereby validating the model within the stochastic framework of the Monte Carlo method. \ref{tab:sigma_comparison} compares the results of these two simulations techniques. For 50'000 particles, the MC method takes 30 minutes whereas the analytical method is independend of the number of particles and takes 1-2 seconds.

\begin{table}[ht]
\centering
\resizebox{\columnwidth}{!}{% Adjusts the table to column width
\begin{tabular}{lcc}
\hline
Distribution & $\sigma$ Analytical (m) & $\sigma$ Monte Carlo (m) \\
\hline
H Non-Scattered & 7.41e-03 & 7.43e-03 \\
H Scattered     & 1.05e-02 & 1.03e-02 \\
V Non-Scattered   & 4.53e-03 & 4.55e-03 \\
V Scattered       & 8.67e-03 & 8.45e-03 \\
\hline
\end{tabular}
}
\caption{Sigma values for analytical and Monte Carlo methods in non-scattered and scattered particle distributions.}
\label{tab:sigma_comparison}
\end{table}


%\subsection{Argument of the current limitation of the air region and the possible improvement gained by adding vacuum chambers to the transfer line}

There is an argument to be made for the current limitation of the air region as at lot of air regions are travelled. The easiest change to implement is to irradiate in IRRAD instead of CHARM which is a few meters upstream. A second change would be the add a vacuum chamber in the switching dipole in F61 which would ease operation.









%%%%%%%%%%%%%%%%%%%%%%%%%%%%%%%%
%%%% SMALL and LARGE OPTICS %%%%
%%%%%%%%%%%%%%%%%%%%%%%%%%%%%%%%
\section{Optics small, large beam}
% This section presents a comparison between the measured beam optics at various points along the transfer line and the predictions made by our MCS-enhanced simulation model.

% Optics measurements were performed using the Beam Television (BTV) monitors along the transfer line. These measurements include profiles at various beam energies, extracted using the slow-extraction process from the CERN PS. Simultaneously, an enhanced model incorporating MCS effects was developed using the MAD-X simulation tool, adjusted to account for the scattering parameters calculated using the following MCS code discussed previously.

% \subsection{Results and Discussion}
% The results from the BTV measurements and MAD-X simulations are compiled in Table \ref{tab:optics_comparison}. The table lists the measured and simulated beam sizes, showing both the horizontal and vertical measurements across different stations.

% % Table of Optics Comparison
% \begin{table}[htb]
% \centering
% \caption{Comparison of Measured and Simulated Beam Sizes with MCS Effects}
% \label{tab:optics_comparison}
% \resizebox{\columnwidth}{!}{% Adjusts the table to column width
% \begin{tabular}{cccc}
% \hline
% \textbf{Station} & \textbf{Measurement (mm)} & \textbf{Simulation (mm)} & \textbf{Difference (\%)} \\
% \hline
% BTV020 & 10.5 & 10.8 & -2.8 \\
% BTV035 & 10.2 & 10.1 & 0.98 \\
% MWPC   & 9.9  & 10.5 & -5.71 \\
% \hline
% \end{tabular}
% }
% \end{table}

% The comparison reveals slight discrepancies between the measured and simulated values, indicative of the challenges associated with precisely modeling MCS effects. Notably, the deviation at the MWPC station suggests a need for further element in the simulation




%\subsection{Comparison with Octavius array}
%\subsection{Comparison with MWPC}

On the last day of the November 2023 CHIMERA HEARTS run, RP did a survey.
Beam energy: 3 GeV/n. Here's a comparison between the MAD-X model and the MWPC.

\begin{figure}[!htb]
   \centering
   \includegraphics*[width=1.0\columnwidth]{rp_survey.png}
   \caption{Difference in beam size between measurement and simulation.}
   \label{fig:diff_beam_size}
\end{figure}

%\subsection{Comparison with BTV}
%\subsection{Mention difficulty of changing the optics because the beam is not well centered}

It was difficult to change the optics because the beam is not centered in the quadrupoles which imparts a dipolar moment.



%\section{Beam size as a function of RFKO gain}
%\subsection{I've observed that the emittance changes at different energy.}
%\subsection{Link with Wesley's contribution}













%%%%%%%%%%%%%%%%%%%%%%%%%%%%%%%%
%%%% ENERGY CONTROL %%%%%%%%%%%%
%%%%%%%%%%%%%%%%%%%%%%%%%%%%%%%%
\section{Energy control}

Rapid adjustment of beam energy is essential for effective radiation testing, as altering the energy spectrum enables the exploration of various Linear Energy Transfer (LET) regions and penetration depths. This methodology, drawn from previous work by the CHIMERA activities at CERN, involves adjusting the magnetic field (B-field) plateau of the Proton Synchrotron (PS) at its flat top. The makerule algorithm calculates the necessary magnetic field in the subsequent magnets of the transfer lines for a corresponding change in rigidity. The removal of the Pole Face Windings (PFW) usage due to their non-linear scaling with beam rigidity, is necessary to provide continuous energy variation.

The 2022 approach of employing distinct cycles for each energy level was replaced in 2023 with a unified cycle. In this cycle, the B-field is adjusted via a Python script. This significantly reduces the time to switch between energies. This system allowed for seamless energy transitions between distinct levels—\SI{650}{}, \SI{750}{}, and \SI{1000}{\mega\electronvolt\per\nucleon}—with the capability to execute these changes rapidly at each new spill, effectively as fast as every \SI{15}-\SI{30}{\second}.

The single cycle has enabled the user to perform energy scans, which are useful to determine precisely the beam’s kinetic energy using its penetration range in a slab of material of a known thickness, e.g. PolyMethyl Methacrylate (PMMA), or to automatize/speed up the SEE test of a component. \cite{noauthor_hearts_nodate}

Python scripting also allows for controlling the fluence—the number of total ions incident on the DUT—which is vital for accurate radiation effect testing. A simple script sends a command to disable the beam once the target fluence is reached. The fluence is measured using the XSEC070, which has been calibrated for flux using the single ion counting capability of the diode.

\subsection{Straggling effects}
The energy range previously mentioned is based on beam energy values inside the PS during the extraction. During the transport in the transfer line, multiple air regions are traversed by the beam, which lowers the beams kinetic energy through energy straggling. This energy straggling at the DUT can be estimated using the FLUKA \cite{battistoni_overview_2015} simulation and measured using the degraders. The FLUKA model also includes other elements that produce straggling effects, such as vacuum windows and beam instrumentation. 

Figure \ref{fig:straggling_effects} illustrates air straggling effects on T8 line ion beams, observed at the T08.BTV35 station. Confirmed by FLUKA simulations, it highlights how the ion beams' exit kinetic energy correlates with their displacement from the center, emphasizing the need for air transport reduction. Installing additional vacuum pipes, particularly in F61.MBXHD025, would improve VHE ion irradiation at IRRAD/CHARM by reducing the straggling effect.
\begin{figure}[!htb]
   \centering
   \includegraphics*[width=1.0\columnwidth]{straggling_effects.png}
   \caption{Measurement of the straggling effects of air in the T8 line at BTV035 compared FLUKA simulation.}
   \label{fig:straggling_effects}
\end{figure}







%%%%%%%%%%%%%%%%%%%%%%%%%%%%%%%%
%%%% CONCLUSION %%%%%%%%%%%%%%%%
%%%%%%%%%%%%%%%%%%%%%%%%%%%%%%%%
\section{CONCLUSION}

The findings from the beam optics studies and simulations for the November 2023 run of the HEARTS experiment underscore the critical importance of understanding and mitigating the effects of air scattering on beam size for the reliable testing of microelectronics destined for space. The comprehensive optics measurements and validated model simulations conducted for the transfer line from the CERN Proton Synchrotron to the CHARM facility offer valuable insights into optimizing beam characteristics for Single Event Effects testing. This work not only highlights the current performance limitations but also sets the stage for future enhancements in radiation testing methodologies at CHARM. By accurately quantifying the impact of air scattering, this contribution paves the way for more effective and reliable testing protocols, ensuring that space-bound microelectronics can withstand the harsh radiation environment of outer space. The integration of MCS effects into beam optics modeling represents a significant step towards more accurate simulations of beam behavior in a real-world setting. While the current model demonstrates a commendable approximation to actual measurements, ongoing adjustments and refinements are necessary to reduce discrepancies and enhance the reliability of the simulation outcomes for future experimental setups. The addition in Xsuite of the xcoll block would also be good for the simulation.

\ifboolexpr{bool{jacowbiblatex}}%
	{\printbibliography}%
	{%
	% "biblatex" is not used, go the "manual" way
	
	%\begin{thebibliography}{99}   % Use for  10-99  references
	\begin{thebibliography}{9} % Use for 1-9 references
	
	\bibitem{jacow-help}
		JACoW,
		\url{http://www.jacow.org}
	
	\bibitem{IEEE}
		\textit{IEEE Editorial Style Manual},
		IEEE Periodicals, Piscataway,
		NJ, USA, Oct. 2014, pp. 34--52.

	\bibitem{journal-abbreviations}
	\url{https://woodward.library.ubc.ca/researchhelp/journal-abbreviations/}

	\end{thebibliography}
} % end \ifboolexpr

\end{document}
