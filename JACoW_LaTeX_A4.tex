% !TeX spellcheck = en_GB
% !TeX program = lualatex
%
% v 2.3  Feb 2019   Volker RW Schaa
%		# changes in the collaboration therefore updated file "jacow-collaboration.tex"
%		# all References with DOIs have their period/full stop before the DOI (after pp. or year)
%		# in the author/affiliation block all ZIP codes in square brackets removed as it was not %         understood as optional parameter and ZIP codes had bin put in brackets
%       # References to the current IPAC are changed to "IPAC'19, Melbourne, Australia"
%       # font for ‘url’ style changed to ‘newtxtt’ as it is easier to distinguish "O" and "0"
%
\documentclass[a4paper,
               %boxit,        % check whether paper is inside correct margins
               %titlepage,    % separate title page
               %refpage       % separate references
               %biblatex,     % biblatex is used
               %keeplastbox,   % flushend option: not to un-indent last line in References
               %nospread,     % flushend option: do not fill with whitespace to balance columns
               %hyphens,      % allow \url to hyphenate at "-" (hyphens)
               %xetex,        % use XeLaTeX to process the file
               %luatex,       % use LuaLaTeX to process the file
               ]{jacow}
%
% ONLY FOR \footnote in table/tabular
%
\usepackage{pdfpages,multirow,ragged2e} %
%
% CHANGE SEQUENCE OF GRAPHICS EXTENSION TO BE EMBEDDED
% ----------------------------------------------------
% test for XeTeX where the sequence is by default eps-> pdf, jpg, png, pdf, ...
%    and the JACoW template provides JACpic2v3.eps and JACpic2v3.jpg which
%    might generates errors, therefore PNG and JPG first
%
\makeatletter%
	\ifboolexpr{bool{xetex}}
	 {\renewcommand{\Gin@extensions}{.pdf,%
	                    .png,.jpg,.bmp,.pict,.tif,.psd,.mac,.sga,.tga,.gif,%
	                    .eps,.ps,%
	                    }}{}
\makeatother

% CHECK FOR XeTeX/LuaTeX BEFORE DEFINING AN INPUT ENCODING
% --------------------------------------------------------
%   utf8  is default for XeTeX/LuaTeX
%   utf8  in LaTeX only realises a small portion of codes
%
\ifboolexpr{bool{xetex} or bool{luatex}} % test for XeTeX/LuaTeX
 {}                                      % input encoding is utf8 by default
 {\usepackage[utf8]{inputenc}}           % switch to utf8

\usepackage[USenglish]{babel}

\DeclareSIUnit\nucleon{u}

%
% if BibLaTeX is used
%
\ifboolexpr{bool{jacowbiblatex}}%
 {%
  \addbibresource{jacow-test.bib}
  \addbibresource{biblatex-examples.bib}
 }{}
\listfiles

%%
%%   Lengths for the spaces in the title
%%   \setlength\titleblockstartskip{..}  %before title, default 3pt
%%   \setlength\titleblockmiddleskip{..} %between title + author, default 1em
%%   \setlength\titleblockendskip{..}    %afterauthor, default 1em

\begin{document}

\title{Beam Optics Modelling of Slow-Extracted Very High-Energy Heavy Ions from the CERN Proton Synchrotron for Radiation Effects Testing \thanks{The HEARTS project is funded by the European Union under Grant Agreement No. 101082402, through the Space Work Programme of the European Commission.}}

\author{E. P. Johnson\thanks{eliott.philippe.johnson@cern.ch}, M.A. Fraser, CERN, Geneva, CH \\
		P. Contributor\textsuperscript{1}, Name of Institute or Affiliation, City, Country \\
		\textsuperscript{1}also at Name of Secondary Institute or Affiliation, City, Country}
	
\maketitle

%
\begin{abstract}
   Testing of space-bound microelectronics plays a crucial role in ensuring the reliability of electronics exposed to the challenging radiation environment of outer space. This contribution describes the beam optics studies carried out for the run held in November 2023 in the context of the CERN High-Energy Accelerators for Radiation Testing and Shielding (HEARTS) experiment. It also delves into an investigation of the initial conditions at the start of the transfer line from the CERN Proton Synchrotron to the CERN High Energy Accelerator Mixed-field (CHARM) facility. Comprehensive optics measurement and simulation campaigns were carried out for this purpose and are presented here. Using a validated optics model of the transfer line, the impact of air scattering on the beam size was quantified with MAD-X and FLUKA, providing valuable insights into the current performance and limitations for Single Event Effects (SEE) testing at CHARM.
\end{abstract}



\section{OPTICS MEASUREMENTS}

The HEARTS activities take place in the East Area of CERN and use the slow extracted beam from the PS through the F61 and T8 transfer lines to irradiate the DUT (device under test) in CHARM (Cern High energy AcceleRator Mixed field facility).

\subsection{Quadrupole scan measurement}

Prior to the ion commissioning quadrupole scan were performed on the East Dump. The extraction line from the PS to the East Dump has three quadrupoles that can be scanned and a btv to perform beam size measurements. The quadrupole scan was performed in the high energy range of the ion activites for HEARTS to diminish the effects of beam size scattering. We used the Pb (lead) ion beam at \SI{2.0}{\giga\electronvolt\per\nucleon} and 11 different quadrupoles scans to match the initial conditions of the F61 to the East Dump. Which gave an initial set of conditions where we matched on $\beta_{x,y}$, $\alpha_{x,y}$ Courant–Snyder (Twiss) parameters and $\epsilon_{n,x,y}$ the normalized emittances.

This lead to this set of initial conditions

\begin{table}[h!]
    \centering
    \caption{Matched Initial Parameters as of October 13, 2023}
    \label{tab:initial_conditions}
    \begin{tabular}{l c c}
        \hline
        Parameter & Horizontal & Vertical \\
        \hline
        $\beta_0$ (m) & 53.074 & 3.675 \\
        $\alpha_0$ & -13.191 & 0.859 \\
        $D_0$ (m) & 0.13 & 0.0 \\
        $D'_0$ & 0.02 & 0.0 \\
        $\varepsilon_{n}$ ($\text{m}^{-5}$) & $2.53 \times 10^{-5}$ & $6.94 \times 10^{-6}$ \\
        \hline
        \multicolumn{2}{l}{\textit{Other Parameters}} \\
        \hline
        $\sigma_e$ & \multicolumn{2}{c}{0.0045} \\
        \hline
    \end{tabular}
\end{table}


\subsection{Dispersion measurement technique and result}


The dispersion was measured in T8 at three different locations: T08.BTV020, T08.BTV035, and T08.MWPC, which are respectively three Beam Television (BTV) monitors and the Multi-Wire Proportional Chamber (MWPC). The procedure was to change the momentum of the beam through the revolution frequency and measure the change in the centroid of the beam at these observation points.

The dispersion is given by the formula:

\begin{equation}
D = \frac{\Delta x}{\frac{\Delta p}{p}}
\end{equation}

where $\Delta x$ is the change in the centroid position of the beam, and $\frac{\Delta p}{p}$ is the relative change in momentum of the beam.

The relationship between the change in revolution frequency and the relative change in beam momentum is expressed as:

\begin{equation}
\frac{\Delta f}{f} = -\eta \frac{\Delta p}{p}
\end{equation}

Here, $\eta$ is the slip factor, defined by:

\begin{equation}
\eta = \left(\frac{1}{\gamma_{\text{tr}}^{2}} - \frac{1}{\gamma^{2}}\right)
\end{equation}

where $\gamma$ is the relativistic Lorentz factor for the beam, and $\gamma_{\text{tr}}$ is the transition gamma. The values were found using the energy of the ion beam and a MAD-X model of the PS where $\gamma = 11.414$ and $\gamma_{\text{tr}} = 6.12719$. The slip factor, $\eta$, can be calculated using these values.

The revolution frequency, $f_{\text{rev}_0}$, is initially $452.0$ kHz. The dispersion value, $D$, can be calculated from the slope of the measured change in centroid position versus the relative change in momentum, corrected by the factor of $\frac{1}{\eta f_{\text{rev}_0}}$.



\begin{figure}[!htb]
   \centering
   \includegraphics*[width=1.0\columnwidth]{beam_size_diff.png}
   \caption{Difference in beam size between measurement and simulation.}
   \label{fig:diff_beam_size}
\end{figure}

Here's a table of the dispersion measurements:

\begin{table}[h!]
\centering
\caption{Dispersion Measurements}
\begin{tabular}{l c c}
\hline
Device & \(Dx\) & \(Dy\) \\
\hline
BTV35  & \(0.851 \pm 0.047\) & \(-0.061 \pm 0.009\) \\
BTV20  & \(-0.197 \pm 0.034\) & \(0.035 \pm 0.031\) \\
MWPC   & \(-0.963 \pm 0.082\) & \(-0.169 \pm 0.019\) \\
\hline
\end{tabular}
\end{table}



\begin{figure}[!htb]
   \centering
   \includegraphics*[width=1.0\columnwidth]{dispersion_diff.png}
   \caption{Difference in dispersion before and after the rematch.}
   \label{fig:dispersion}
\end{figure}

I found some dispersion values that I then compare to the model. Then I rematch on the dx, dpx, dy, dpy and used these guys as initial conditions.

This is not exactly correct because I the dispersion steering only for one optic. What I should have done is do the frev for every optics.

\begin{table}[h!]
    \centering
    \caption{Matched Initial Parameters as of April 10, 2024}
    \label{tab:initial_conditions_2024}
    \begin{tabular}{l c c}
        \hline
        Parameter & Horizontal & Vertical \\
        \hline
        $\beta_0$ (m) & 66.748 & 3.764 \\
        $\alpha_0$ & -16.272 & 0.703 \\
        $D_0$ (m) & 0.086 & -0.003 \\
        $D'_0$ & 0.017 & -0.005 \\
        $\varepsilon_{n}$ ($\text{m}^{-5}$) & $2.28 \times 10^{-5}$ & $8.63 \times 10^{-6}$ \\
        \hline
        \multicolumn{2}{l}{\textit{Other Parameters}} \\
        \hline
        $\sigma_e$ & \multicolumn{2}{c}{0.0045} \\
        \hline
    \end{tabular}
\end{table}


\subsection{Pybobyqa to fit the model to the measurement}
\subsection{Kick response}
\subsection{Initial conditions}
\subsection{Challenges with stray field, motivates why measure initial conditions}

\section{Optics small, large beam}
\subsection{Comparison with Octavius array}
\subsection{Comparison with MWPC}
\subsection{Comparison with BTV}
\subsection{Mention difficulty of changing the optics because the beam is not well centered}
\subsection{Upload to YASP}
\subsection{Measurement of beam size along the line using the BTVs, MWPC and IRRAD BPMs}
\subsection{Comparison with model}

\section{Beam size as a function of RFKO gain}
\subsection{I've observed that the emittance changes at different energy.}
\subsection{Link with Wesley's contribution}

\section{Air scattering}
\subsection{Overview of the formula}
\subsection{Comparison with FLUKA}
\subsection{Comparison with Xsuite}
\subsection{Argument of the current limitation of the air region and the possible imrovement gained by adding vacuum chambers to the transfer line}
\subsection{Straggling effects}

\section{Energy scan}
\subsection{How the energy is controlled}
\subsection{Fluence controlled}


\section{CONCLUSION}



%
% only for "biblatex"
%
\ifboolexpr{bool{jacowbiblatex}}%
	{\printbibliography}%
	{%
	% "biblatex" is not used, go the "manual" way
	
	%\begin{thebibliography}{99}   % Use for  10-99  references
	\begin{thebibliography}{9} % Use for 1-9 references
	
	\bibitem{jacow-help}
		JACoW,
		\url{http://www.jacow.org}
	
	\bibitem{IEEE}
		\textit{IEEE Editorial Style Manual},
		IEEE Periodicals, Piscataway,
		NJ, USA, Oct. 2014, pp. 34--52.

	\bibitem{journal-abbreviations}
	\url{https://woodward.library.ubc.ca/researchhelp/journal-abbreviations/}

	\end{thebibliography}
} % end \ifboolexpr

\end{document}
