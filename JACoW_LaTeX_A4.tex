% !TeX spellcheck = en_GB
% !TeX program = lualatex
%
% v 2.3  Feb 2019   Volker RW Schaa
%		# changes in the collaboration therefore updated file "jacow-collaboration.tex"
%		# all References with DOIs have their period/full stop before the DOI (after pp. or year)
%		# in the author/affiliation block all ZIP codes in square brackets removed as it was not %         understood as optional parameter and ZIP codes had bin put in brackets
%       # References to the current IPAC are changed to "IPAC'19, Melbourne, Australia"
%       # font for ‘url’ style changed to ‘newtxtt’ as it is easier to distinguish "O" and "0"
%
\documentclass[a4paper,
               %boxit,        % check whether paper is inside correct margins
               %titlepage,    % separate title page
               %refpage       % separate references
               biblatex,     % biblatex is used
               %keeplastbox,   % flushend option: not to un-indent last line in References
               %nospread,     % flushend option: do not fill with whitespace to balance columns
               %hyphens,      % allow \url to hyphenate at "-" (hyphens)
               %xetex,        % use XeLaTeX to process the file
               %luatex,       % use LuaLaTeX to process the file
               ]{jacow}
%
% ONLY FOR \footnote in table/tabular
%
\usepackage{pdfpages,multirow,ragged2e} %
%
% CHANGE SEQUENCE OF GRAPHICS EXTENSION TO BE EMBEDDED
% ----------------------------------------------------
% test for XeTeX where the sequence is by default eps-> pdf, jpg, png, pdf, ...
%    and the JACoW template provides JACpic2v3.eps and JACpic2v3.jpg which
%    might generates errors, therefore PNG and JPG first
%
\makeatletter%
	\ifboolexpr{bool{xetex}}
	 {\renewcommand{\Gin@extensions}{.pdf,%
	                    .png,.jpg,.bmp,.pict,.tif,.psd,.mac,.sga,.tga,.gif,%
	                    .eps,.ps,%
	                    }}{}
\makeatother

% CHECK FOR XeTeX/LuaTeX BEFORE DEFINING AN INPUT ENCODING
% --------------------------------------------------------
%   utf8  is default for XeTeX/LuaTeX
%   utf8  in LaTeX only realises a small portion of codes
%
\ifboolexpr{bool{xetex} or bool{luatex}} % test for XeTeX/LuaTeX
 {}                                      % input encoding is utf8 by default
 {\usepackage[utf8]{inputenc}}           % switch to utf8

\usepackage[USenglish]{babel}
\usepackage{hyperref}

\DeclareSIUnit\nucleon{u}

%
% if BibLaTeX is used
%
\ifboolexpr{bool{jacowbiblatex}}%
 {%
  \addbibresource{references.bib}
 }{}



%%
%%   Lengths for the spaces in the title
%%   \setlength\titleblockstartskip{..}  %before title, default 3pt
%%   \setlength\titleblockmiddleskip{..} %between title + author, default 1em
%%   \setlength\titleblockendskip{..}    %afterauthor, default 1em

\begin{document}

\title{Beam Optics Modelling of Slow-Extracted Very High-Energy Heavy Ions from the CERN Proton Synchrotron for Radiation Effects Testing \thanks{The HEARTS project is funded by the European Union under Grant Agreement No. 101082402, through the Space Work Programme of the European Commission.}}

\author{E. P. Johnson\thanks{eliott.philippe.johnson@cern.ch}, M.A. Fraser, CERN, Geneva, CH \\
		P. Contributor\textsuperscript{1}, Name of Institute or Affiliation, City, Country \\
		\textsuperscript{1}also at Name of Secondary Institute or Affiliation, City, Country}
	
\maketitle

%
\begin{abstract}
   Testing of space-bound microelectronics plays a crucial role in ensuring the reliability of electronics exposed to the challenging radiation environment of outer space. This contribution describes the beam optics studies carried out for the run held in November 2023 in the context of the CERN High-Energy Accelerators for Radiation Testing and Shielding (HEARTS) experiment. It also delves into an investigation of the initial conditions at the start of the transfer line from the CERN Proton Synchrotron to the CERN High Energy Accelerator Mixed-field (CHARM) facility. Comprehensive optics measurement and simulation campaigns were carried out for this purpose and are presented here. Using a validated optics model of the transfer line, the impact of air scattering on the beam size was quantified with MAD-X and FLUKA, providing valuable insights into the current performance and limitations for Single Event Effects (SEE) testing at CHARM.
\end{abstract}



\section{OPTICS MEASUREMENTS}

The HEARTS activities take place in the East Area of CERN and use the slow extracted beam from the PS through the F61 and T8 transfer lines to irradiate the DUT (device under test) in CHARM (Cern High energy AcceleRator Mixed field facility).

\subsection{Quadrupole scan measurement}

Prior to the ion commissioning quadrupole scan were performed on the East Dump. The extraction line from the PS to the East Dump has three quadrupoles that can be scanned and a btv to perform beam size measurements. The quadrupole scan was performed in the high energy range of the ion activites for HEARTS to diminish the effects of beam size scattering. We used the Pb (lead) ion beam at \SI{2.0}{\giga\electronvolt\per\nucleon} and 11 different quadrupoles scans to match the initial conditions of the F61 to the East Dump. Which gave an initial set of conditions where we matched on $\beta_{x,y}$, $\alpha_{x,y}$ Courant–Snyder (Twiss) parameters and $\epsilon_{n,x,y}$ the normalized emittances.

This lead to a first set of initial conditions. Of course, in the beam size formula, the beta and the dispersion are linked.



\subsection{Dispersion measurement technique and result}


The dispersion was measured in T8 at three different locations: T08.BTV020, T08.BTV035, and T08.MWPC, which are respectively three Beam Television (BTV) monitors and the Multi-Wire Proportional Chamber (MWPC). The procedure was to change the momentum of the beam through the revolution frequency and measure the change in the centroid of the beam at these observation points.

The dispersion is given by the formula:

\begin{equation}
D = \frac{\Delta x}{\frac{\Delta p}{p}}
\end{equation}

where $\Delta x$ is the change in the centroid position of the beam, and $\frac{\Delta p}{p}$ is the relative change in momentum of the beam.

The relationship between the change in revolution frequency and the relative change in beam momentum is expressed as:

\begin{equation}
\frac{\Delta f}{f} = -\eta \frac{\Delta p}{p}
\end{equation}

Here, $\eta$ is the slip factor, defined by:

\begin{equation}
\eta = \left(\frac{1}{\gamma_{\text{tr}}^{2}} - \frac{1}{\gamma^{2}}\right)
\end{equation}

where $\gamma$ is the relativistic Lorentz factor for the beam, and $\gamma_{\text{tr}}$ is the transition gamma. The values were found using the energy of the ion beam and a MAD-X model of the PS where $\gamma = 11.414$ and $\gamma_{\text{tr}} = 6.12719$. The slip factor, $\eta$, can be calculated using these values.

The revolution frequency, $f_{\text{rev}_0}$, is initially $452.0$ kHz. The dispersion value, $D$, can be calculated from the slope of the measured change in centroid position versus the relative change in momentum, corrected by the factor of $\frac{1}{\eta f_{\text{rev}_0}}$.



\begin{figure}[!htb]
   \centering
   \includegraphics*[width=1.0\columnwidth]{beam_size_diff.png}
   \caption{Difference in beam size between measurement and simulation.}
   \label{fig:diff_beam_size}
\end{figure}

Here's a table of the dispersion measurements:

\begin{table}[h!]
\centering
\caption{Dispersion Measurements}
\begin{tabular}{l c c}
\hline
Device & \(Dx\) & \(Dy\) \\
\hline
BTV35  & \(0.851 \pm 0.047\) & \(-0.061 \pm 0.009\) \\
BTV20  & \(-0.197 \pm 0.034\) & \(0.035 \pm 0.031\) \\
MWPC   & \(-0.963 \pm 0.082\) & \(-0.169 \pm 0.019\) \\
\hline
\end{tabular}
\end{table}



\begin{figure}[!htb]
   \centering
   \includegraphics*[width=1.0\columnwidth]{dispersion_diff.png}
   \caption{Difference in dispersion before and after the rematch.}
   \label{fig:dispersion}
\end{figure}

I found some dispersion values that I then compare to the model. Then I rematch on the dx, dpx, dy, dpy and used these guys as initial conditions. Because the twiss parameters betas are linked to the dispersion in the beam size formula, these will compensate for the change in the initial dispersion.

This is not exactly correct because I the dispersion steering only for one optic. What I should have done is do the frev for every optics.

\begin{table}[h!]
    \centering
    \caption{Comparison of Matched Initial Parameters}
    \label{tab:initial_conditions_comparison_single_col}
    \begin{tabular}{l c}
        \hline
        Parameter & Value Change \\
        \hline
        $\beta_x$ (m) & 53.074 \(\rightarrow\) 66.748 \\
        $\beta_y$ (m) & 3.675 \(\rightarrow\) 3.764 \\
        $\alpha_x$ & -13.191 \(\rightarrow\) -16.272 \\
        $\alpha_y$ & 0.859 \(\rightarrow\) 0.703 \\
        $D_x$ (m) & 0.13 \(\rightarrow\) 0.086 \\
        $D_y$ (m) & 0.0 \(\rightarrow\) -0.003 \\
        $D'_x$ & 0.02 \(\rightarrow\) 0.017 \\
        $D'_y$ & 0.0 \(\rightarrow\) -0.005 \\
        $\varepsilon_{nx}$ ($\text{m}^{-5}$) & $2.53 \times 10^{-5}$ \(\rightarrow\) $2.28 \times 10^{-5}$ \\
        $\varepsilon_{ny}$ ($\text{m}^{-5}$) & $6.94 \times 10^{-6}$ \(\rightarrow\) $8.63 \times 10^{-6}$ \\
        \hline
        \textit{Other Parameters} & \\
        \hline
        $\sigma_e$ & 0.0045 (Unchanged) \\
        \hline
    \end{tabular}
\end{table}

\subsection{Pybobyqa to fit the model to the measurement}
\subsection{Kick response}
\subsection{Initial conditions}
\subsection{Challenges with stray field, motivates why measure initial conditions}

\section{Optics small, large beam}
\subsection{Comparison with Octavius array}
\subsection{Comparison with MWPC}

On the last day of the November 2023 CHIMERA HEARTS run, RP did a survey.
Beam energy: 3 GeV/n. Here's a comparison between the MAD-X model and the MWPC.

\begin{figure}[!htb]
   \centering
   \includegraphics*[width=1.0\columnwidth]{rp_survey.png}
   \caption{Difference in beam size between measurement and simulation.}
   \label{fig:diff_beam_size}
\end{figure}

\subsection{Comparison with BTV}
\subsection{Mention difficulty of changing the optics because the beam is not well centered}
\subsection{Upload to YASP}
\subsection{Measurement of beam size along the line using the BTVs, MWPC and IRRAD BPMs}
\subsection{Comparison with model}

\section{Beam size as a function of RFKO gain}
\subsection{I've observed that the emittance changes at different energy.}
\subsection{Link with Wesley's contribution}

\section{Air scattering}

In particle physics, accurately predicting beam size in accelerators is essential but challenging due to multi-Coulomb scattering. This phenomenon, where charged particles interact with atomic nuclei, significantly alters beam trajectory and spread, deviating from idealized simulations. Ignoring these interactions leads to inaccuracies in beam focus and efficiency, impacting experiments and applications like medical therapies. Incorporating multi-Coulomb scattering into simulations enhances the precision of beam behavior predictions, facilitating better experiment design and accelerator functionality.

\subsection{Overview of the formula}

The fundamental formula that describes the angle of deflection due to multi-Coulomb scattering as a particle passes through a material is given by the following formula:

\[
\theta_{rms} = \frac{13.6 \text{MeV/c}}{p\beta_{p}}q_{p}\sqrt\frac{L}{L_{rad}}\]

where:
\begin{itemize}
\item p = $24\cdot 10^{3}$ Beam total energy in MeV
\item q = 1
\item P = 1.01325  Standard air pressure at sea level in Bar
\item P\_Torr = P x 750.062 Standard air pressure at sea level in Torr
\item L\_rad0 = 301 for air \href{https://cds.cern.ch/record/941314/files/p245.pdf}{Table with radiation lengths}
\item L\_rad = L\_rad0/(P\_Torr/760)
\end{itemize}

\[
\epsilon_{1} = \epsilon_{0}\sqrt{1+\frac{\beta_{0}\theta_{rms}^{2}}{\epsilon_{0}}}
\]

\[
\beta_{1}=\frac{\beta_{0}}{\sqrt{1+\frac{\beta_{0}\theta_{rms}^{2}}{\epsilon_{0}}}}
\]


\subsection{Overview of the script}

The concept behind the script is designed to handle multi-coulomb interactions by integrating a straightforward debugging process with optimization through pybobqa. It introduces a user interaction mechanism where the user defines the boundaries of an air region with markers "AIR\_START" and "AIR\_END," alongside a final "END" marker. Additionally, users can specify the step size for computations, which can vary from 1cm to 1m or any other user-defined measure. The core functionality of the script includes adding inner markers within these air regions at discrete, user-selected intervals for performing calculations. At the onset of the first "AIR\_START" marker, the script is programmed to save the beta function values using "SAVEBETA" and then proceeds to split the sequence at "AIR\_START." Subsequently, it runs a Twiss calculation on this split sequence, utilizing the saved beta function values. However, it updates the beta functions for both the horizontal ( $\beta_{x}$ ) and vertical ( $\beta_{y}$ ) planes, as well as the horizontal and vertical emittances ( $e_{x}$, $e_{y}$ ), based on the calculations. The process continues in a loop: running Twiss calculations up to the next "INNER\_MARKER," saving the beta function values again with "SAVEBETA," and then splitting the sequence, thereby allowing for precise and iterative adjustments based on multi-coulomb interactions within specified air regions.

\begin{figure}[!htb]
   \centering
   \includegraphics*[width=1.0\columnwidth]{air_scattering_1.png}
   \caption{Simple line with simulation of multi-coulomb scattering.}
   \label{fig:simple_line}
\end{figure}

\subsection{Comparison with other simulation tools}

\begin{figure}[!htb]
   \centering
   \includegraphics*[width=1.0\columnwidth]{compare_simulation.png}
   \caption{This plots compare the MAD-X simulation with MCS, Xsuite and FLUKA on the simple line.}
   \label{fig:simple_line}
\end{figure}

\subsection{Argument of the current limitation of the air region and the possible imrovement gained by adding vacuum chambers to the transfer line}

\section{Energy control}

Rapid adjustment of beam energy is essential for effective radiation testing, as altering the energy spectrum enables the exploration of various Linear Energy Transfer (LET) regions and penetration depths. This methodology, drawn from previous work by the CHIMERA activities at CERN, involves adjusting the magnetic field (B-field) plateau of the Proton Synchrotron (PS) at its flat top. The makerule algorithm calculates the necessary magnetic field in the subsequent magnets of the transfer lines for a corresponding change in rigidity. The removal of the Pole Face Windings (PFW) usage due to their non-linear scaling with beam rigidity, is necessary to provide continuous energy variation.

The 2022 approach of employing distinct cycles for each energy level was replaced in 2023 with a unified cycle. In this cycle, the B-field is adjusted via a Python script. This significantly reduces the time to switch between energies. This system allowed for seamless energy transitions between distinct levels—\SI{650}{}, \SI{750}{}, and \SI{1000}{\mega\electronvolt\per\nucleon}—with the capability to execute these changes rapidly at each new spill, effectively as fast as every \SI{15}-\SI{30}{\second}.

The single cycle has enabled the user to perform energy scans, which are useful to determine precisely the beam’s kinetic energy using its penetration range in a slab of material of a known thickness, e.g. PolyMethyl Methacrylate (PMMA), or to automatize/speed up the SEE test of a component. The energy range previously mentioned is based on beam energy values inside the PS during the extraction. During the transport in the transfer line, multiple air regions are traversed by the beam, which lowers the beams kinetic energy through energy straggling. This energy straggling at the DUT can be estimated using the FLUKA [2] simulation and measured using the degraders. The FLUKA model also includes other elements that produce straggling effects, such as vacuum windows and beam instrumentation. \cite{noauthor_hearts_nodate}

Controlling the fluence, the number of particles incident per unit area, is vital for accurate radiation effect testing. The revised energy control mechanism enhances the ability to adjust the fluence quickly and accurately across different energy levels. This capability is essential for simulating the diverse radiation environments encountered in space, providing a more robust assessment of microelectronics' resilience under various conditions.

This enhanced energy scan technique, featuring rapid and precise energy adjustments, represents a significant advancement in the domain of radiation effects testing. It enables more efficient and versatile testing procedures, vital for evaluating the durability of space-bound microelectronics against the spectrum of radiation threats in outer space.

\subsection{Straggling effects}

The straggling effects of air on ion beams in the T8 line, as observed at the T08.BTV35 monitoring station. The data, corroborated by FLUKA simulations, showcases the relationship between the exit kinetic energy of the ion beams and their measured displacement from the beam center. This plot, seen in Figure \ref{fig:straggling_effects}, serves as a pivotal illustration of the challenges of low energy ion transport. It underscores the necessity for reduces the transport through air. By mitigating air straggling, we aim to achieve a more consistent beam trajectory, thus enhancing operational capabilities for VHE ion irradiation at the IRRAD/CHARM facility. The installation of an additional vacuum pipe in F61.MBXHD025 would allow for the linear scaling of magnet strengths, ensuring precise control over the beam path and maintaining beam integrity at varying energies and ion species.
\begin{figure}[!htb]
   \centering
   \includegraphics*[width=1.0\columnwidth]{straggling_effects.png}
   \caption{This plots shows the straggling effects of air in the T8 line.}
   \label{fig:straggling_effects}
\end{figure}



\section{CONCLUSION}

The findings from the beam optics studies and simulations for the November 2023 run of the HEARTS experiment underscore the critical importance of understanding and mitigating the effects of air scattering on beam size for the reliable testing of microelectronics destined for space. The comprehensive optics measurements and validated model simulations conducted for the transfer line from the CERN Proton Synchrotron to the CHARM facility offer valuable insights into optimizing beam characteristics for Single Event Effects testing. This work not only highlights the current performance limitations but also sets the stage for future enhancements in radiation testing methodologies at CHARM. By accurately quantifying the impact of air scattering, this contribution paves the way for more effective and reliable testing protocols, ensuring that space-bound microelectronics can withstand the harsh radiation environment of outer space.

\ifboolexpr{bool{jacowbiblatex}}%
	{\printbibliography}%
	{%
	% "biblatex" is not used, go the "manual" way
	
	%\begin{thebibliography}{99}   % Use for  10-99  references
	\begin{thebibliography}{9} % Use for 1-9 references
	
	\bibitem{jacow-help}
		JACoW,
		\url{http://www.jacow.org}
	
	\bibitem{IEEE}
		\textit{IEEE Editorial Style Manual},
		IEEE Periodicals, Piscataway,
		NJ, USA, Oct. 2014, pp. 34--52.

	\bibitem{journal-abbreviations}
	\url{https://woodward.library.ubc.ca/researchhelp/journal-abbreviations/}

	\end{thebibliography}
} % end \ifboolexpr

\end{document}
